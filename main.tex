\documentclass{article}
\usepackage{graphicx} % Required for inserting images
\usepackage[L7x,T1]{fontenc}
\usepackage[utf8]{inputenc}
\usepackage{amsmath}
\usepackage{multicol}

\title{\bold{IBM System 370 vs. Raytheon 704} \\1970 metų architektūros}
\author{Rasa Šileikytė}
\date{Gruodis 2024}

\begin{document}

\maketitle

\section*{Elementinės kompiuterių bazės bei matmenys}
\begin{multicols}{2}
IBM System 370 (Modelis 145):
\begin{itemize}
\item Bipoliniai integriniai grandynai, didelis integracijos mąstas (LSI).
\item Pagrindinė atmintis - puslaidininkiai (Pirmas toks IBM kompiuteris/minikompiuteris)
\item Dydis - Didelės apimties (nedidelio kambario dydžio)
\item Svoris - apie toną.
\item Energijos suvartojimas - didelis
\end{itemize}

\columnbreak

Raytheon 704 :
\begin{itemize}
\item Atmintis - Magnetinės šerdys
\item Dydis - Nedidelės apimties (16x18x24)(coliai)
\item Svoris - 75(lbs) arba apie 34 kg.
\item Energijos suvartojimas - mažesnis (palyginus su IBM)
\end{itemize}
\end{multicols}



\end{document}
