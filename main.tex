\documentclass{article}
\usepackage{graphicx} % Required for inserting images
\usepackage[L7x,T1]{fontenc}
\usepackage[utf8]{inputenc}
\usepackage{amsmath}
\usepackage{multicol}

\title{\bold{IBM System 370 vs. Raytheon 704} \\1970 metų architektūros}
\author{Rasa Šileikytė}
\date{Gruodis 2024}

\begin{document}

\maketitle

\section*{Elementinės kompiuterių bazės bei matmenys}
\begin{multicols}{2}
IBM System 370 (Modelis 145):
\begin{itemize}
\item Bipoliniai integriniai grandynai, didelis integracijos mąstas (LSI).
\item Pagrindinė atmintis - puslaidininkiai (Pirmas toks IBM kompiuteris/minikompiuteris)
\item Dydis - Didelės apimties (nedidelio kambario dydžio)
\item Svoris - apie toną.
\item Energijos suvartojimas - didelis
\end{itemize}

\columnbreak

Raytheon 704 :
\begin{itemize}
\item Atmintis - Magnetinės šerdys
\item Dydis - Nedidelės apimties (16x18x24)(coliai)
\item Svoris - 75(lbs) arba apie 34 kg.
\item Energijos suvartojimas - mažas (palyginus su IBM)
\end{itemize}
\end{multicols}
\section*{Architektūros įpatumai}
\begin{multicols}{2}
IBM System 370 (Modelis 145): registrinis - šešiolika 32-bitų bendros paskirties registrų (indeksavimas, akumuliatoriai), keturi 64-bitų float tipo registrai. Naudojami požymių bitai (Zero, Negative, Carry, Overflow). Mašininis žodis 32-bitų.

\columnbreak
Raytheon 704 : akumuliatorinis ir registrinis - atminties dydis žodžiais 4096, žodžio ilgis 16-bitų ir 1 baitas parity, keturi 15-bitų adresų registrai ir vienas indeksavimo registras. Naudojami požymių bitai (Negative, Zero) Mašininis žodis 16 bitų.
\end{multicols}

\section*{Architektų instrukcijos}

IBM System 370 (Modelis 145): Adresų erdvė ištisinė, efektyvus adreso plotis 32-bitai, maksimalus atminties kiekis 4 GB, tipiškas atminties kiekis apie 512 MB. Ši sistema palaikė virtualiąją atmintį realizuotą puslapiavimu. Instrukcijų klasės:
\begin{itemize}
    \item Aritmetinės ir loginės komandos (sudėtis, atimtis ir t.t)
    \item Atminties manipuliavimo komandos (LDA - load accumulator ir t.t)
    \item Kontrolės ir šokimo komandos (BR - branch)
    \item Input/Output komandos
    \item Atminties valdymo komandos (komandos atminties suskirstymui)
\end{itemize}
Instrukcijų formatai:
\begin{itemize}
    \item Op kodas (Opcode): 8 bitai.
    \item Adresavimo operandai: Iki 24 bitų, priklausomai nuo instrukcijos tipo.
\end{itemize}
Instrukcijų pavyzdžiai:
\begin{itemize}
    \item LDA (Load Accumulator) – Įkrauna duomenis į akumuliatorių.
    \item STA (Store Accumulator) – Išsaugo duomenis iš akumuliatoriaus į atmintį.
    \item ADD (Add) – Atlieka sudėjimo operaciją su registru.
    \item SUB (Subtract) – Atlieka atimties operaciją.
    \item AND (Logical AND) – Atlieka loginį AND.
    \item OR (Logical OR) – Atlieka loginį OR.
    \item BR (Branch) – Atlikti šuolį į kitą vietą atmintyje.
    \item BNE (Branch if Not Equal) – Šokti, jei sąlyga įvykdyta.
    \item JUMP (Unconditional jump) – Ne sąlyginis šuolis.
    \item TST (Test) – Testuoja bitų būseną.
    \item CMP (Compare) – Palygina du reikšmes.
    \item MUL (Multiply) – Atlieka dauginimo operaciją.
    \item DIV (Divide) – Atlieka dalijimo operaciją.
    \item SVC (Supervisor Call) – Kreipiasi į operacinę sistemą.

\end{itemize}



Raytheon 704 : Atmintis ištisinė (kaupiama moduliuose), efektyvus adreso plotis 16-bitų, maksimalus atminties kiekis 16 MB, tipiškas atminties kiekis apie 8 MB. Ši sistema nepalaikė virtualios atminties vykdė tiesioginį adresavimą.
Instrukcijų klasės:
\begin{itemize}
    \item Aritmetinės ir loginės komandos (sudėtis, atimtis ir t.t)
     \item Duomenų perkėlimo komandos (LDA - load Accumulator ir t.t)
    \item Kontrolės ir šokimo komandos (BR - branch)
    \item Specializuotos komandos (nulinio reikšmės testavimas).
\end{itemize}
Instrukcijų formatai:
\begin{itemize}
    \item Op kodas: 6 bitai.
    \item Adresavimo operandai: 18 bitų, atitinkantys duomenų adresus ar reikšmes.
\end{itemize}
Instrukcijų pavyzdžiai:
\begin{itemize}
    \item ADD (Add) – Atlieka sudėjimo operaciją.
    \item SUB (Subtract) – Atlieka atimties operaciją.
    \item LDA (Load Accumulator) – Įkrauna reikšmę į akumuliatorių.
    \item STA (Store Accumulator) – Saugo reikšmę iš akumuliatoriaus į atmintį.
    \item CMP (Compare) – Palygina du registrus.
    \item BR (Branch) – Atlikti šuolį.
    \item BNE (Branch if Not Equal) – Šokti, jeigu sąlyga įvykdyta.
    \item JSR (Jump to Subroutine) – Atlikti šuolį į subrutinę.
    \item TST (Test) – Tikrina duomenų būseną.
    \item NOP (No Operation) – Neatlieka jokios operacijos (laiko švaistymas).
    \item MUL (Multiply) – Atlieka dauginimo operaciją.
    \item DIV (Divide) – Atlieka dalijimo operaciją.
    \item DIVU (Divide Unsigned) – Atlieka nesigned dalijimą.
\end{itemize}
\section*{Adresavimo būdai}
IBM System/370 :
\begin{itemize}
    \item Registrų adresavimas (Register Addressing)
    \item Tiesinis adresavimas (Imediate Addressing)
    \item Tiesioginis adresavimas (Direct Addressing)
    \item Ne tiesioginis adresavimas (Indirect Addressing)
    \item Ne tiesioginis su viena indekso reikšme (Indexed Addressing)
    \item Adresavimas per bazinius registrus (Base Register Addressing)
    \item Ne tiesioginis adresavimas su pakeitimu (Relative Addressing)
\end{itemize}
Raytheon 704 :
\begin{itemize}
    \item Registrų adresavimas (Register Addressing)
    \item Tiesioginis adresavimas (Direct Addressing)
    \item Ne tiesioginis adresavimas (Indirect Addressing)
    \item Indekso adresavimas (Indexed Addressing)
    \item Reliatyvus adresavimas (Relative Addressing)
\end{itemize}
Registrų adresavimas, tiesioginis adresavimas, indirektinis adresavimas, indeksuotas adresavimas ir relatyvus adresavimas buvo bendri tiek IBM System/370, tiek Raytheon 704 architektūroms. Abu kompiuteriai palaikė galimybę pasiekti atminties operacijas su registrų ir tiesioginių atminties adresų naudojimu. Raytheon 704 turėjo paprastesnę adresavimo sistemą su mažiau pažangių metodų. IBM System/370 buvo gerokai išvystytas dėl to, kad įtraukė ir atminties valdymo instrukcijas.

\section*{Input/Output galimybės}
IBM System/370 naudojo kanalo valdymo sistemą (angl. Channel I/O) bei Tiesioginė atminties prieiga (DMA) - Kanalo valdymas buvo įtrauktas į procesorių, ir kiekvienas įrenginys buvo prijungtas prie kanalo, kuris buvo atsakingas už duomenų perdavimą tarp įrenginių ir atminties. Ši sistema leido kompiuteriui vykdyti duomenų pervedimo operacijas be tiesioginio procesoriaus įsikišimo. Ši architektūra taip pat palaikė diskinius įrenginius, magnetines juostas, spausdintuvus bei "punch" korteles.
\\
Raytheon 704 taip pat turėjo DMA, leidžiančią tiesiogiai perduoti duomenis tarp I/O įrenginių ir pagrindinės atminties be procesoriaus įsikišimo. Taip pat palaikė diskinius įrenginius, magnetines juostas, spausdintuvus bei "punch" korteles.
\section*{Pertraukimai}
Abi architektūros palaikė I/O pertraukimus ir programinius pertraukimus. Raytheon 704 ir IBM System/370 turėjo laiko pertraukimo mechanizmą, kuris buvo svarbus laiko dalijimo sistemoms.
IBM System/370 turėjo sudėtingesnį ir labiau išvystytą pertraukimo valdymą, įskaitant kanalo pertraukimo mechanizmus ir aukštą prioritetų valdymą, tuo tarpu Raytheon 704 pertraukimo mechanizmai buvo paprastesni, su mažiau sudėtingų prioritetų ir valdymo funkcijų.
IBM System/370 galėjo valdyti daugiau I/O įrenginių.
\section*{Greitaveika}
Taktinis dažnis:
\begin{itemize}
    \item IBM System/370 Model 145: 1 MHz – 2 MHz.
    \item Raytheon 704: 1 MHz.
\end{itemize}
Ciklų skaičius komandai:
\begin{itemize}
    \item IBM System/370 Model 145: 4–10 ciklų.
    \item Raytheon 704: 2–5 ciklai.
\end{itemize}
Vidutinė sistemos greitaveika:
\begin{itemize}
    \item IBM System/370 Model 145 buvo greitesnė ir efektyvesnė sudėtingesniems kompiuteriniams procesams.
    \item Raytheon 704 buvo greitesnė paprastoms užduotims, tačiau turėjo mažesnį našumą, kai kalbama apie sudėtingus skaičiavimus.
\end{itemize}
\section*{Sparčioji atmintis}
IBM System/370 architektūra, ypač Modelis 145, buvo viena pirmųjų, kuri pradėjo naudoti spartinančiąją atmintį. "Cache" atmintis buvo įdiegta kaip spartinimo priemonė, kad sumažintų prieigos laiką prie atminties. Spartinančios atminties dydis buvo ne itin didelis.
\\
Raytheon 704 nenaudojo spartinančiosios atminties, tačiau buvo naudojama specializuota atmintis (registrų rinkiniai), kuri buvo skirta spartinti tam tikras komandas.
\end{document}
